\section{課題4}
\begin{itembox}{課題4}
  表にあるデータを利用する.また潜在的な確率密度分布は正規分布であるとする.P($\omega$i)=1/3 とする.表にあ
  げた各クラスのデータセットは omega1.txt,omega2.txt,omega3.txt である.このとき次の問いに答えよ.

  \begin{enumerate}
    \item テスト点:$(1, 2, 1)^T$, $(5, 3, 2)^T$, $(0, 0, 0)^T$, $(1, 0, 0)^T$ と各クラスの平均との間のマハラノビス距離を求めよ.
    \item これらの点を識別せよ.
    \item 次に P($\omega_1$)=0.8 かつ P($\omega_2$) = P($\omega_3$)=0.1 と仮定し,テスト点をもう一度識別せよ
  \end{enumerate}
\end{itembox}
テスト点:$(1, 2, 1)^T$, $(5, 3, 2)^T$, $(0, 0, 0)^T$, $(1, 0, 0)^T$
に関して, 各クラス集合の平均とのマハラノビス距離

\begin{eqnarray}
  M_{D}(x) = \sqrt{(x - \mu_{i})^T \sum (x - \mu_i)}
\end{eqnarray}

を表\ref{tbl:mahalanobis}に計算した. 

\begin{table}[htbp]
  \begin{small}
    \begin{tabular}{cccc}
      sample points & $\omega_1$ & $\omega_2$ & $\omega_3$ \\
      $(1, 2, 1)^T$ & 21.275575 & 15.5762194 & 9.85157479 \\
      $(5, 3, 2)^T$ & 39.1226698 & 37.4144013 & 9.82164589 \\
      $(0, 0, 0)^T$ & 9.175011 & 4.78336417 & 23.4702917 \\
      $(1, 0, 0)^T$ & 12.0187365 & 9.85778568 & 20.7996021
    \end{tabular}
    \caption{テスト点の各クラス集合の平均とのマハラノビス距離}
    \label{tbl:mahalanobis}
  \end{small}
\end{table}

確率的生成モデルを用いて, これらのテスト点を識別したところ, 
表\ref{tbl:prob-discrimination-1}に示す識別結果となった. ($P(\omega_i)$ = 1/3)

$P(\omega_1)=0.8$, $P(\omega_2)=P(\omega_3)=0.1$として識別を行ったところ
表\ref{tbl:prob-discrimination-2}に示す識別結果となり, 
表\ref{tbl:prob-discrimination-1}と同じ結果となった. 

\begin{table}[htbp]
  \begin{tabular}{cccc}
    % test point 1 & test point 2 & test point 3 & test point 4 \\
    $(1, 2, 1)^T$ & $(5, 3, 2)^T$ & $(0, 0, 0)^T$ & $(1, 0, 0)^T$ \\
    $\omega_2$ & $\omega_2$ & $\omega_1$ & $\omega_1$
  \end{tabular}
  \caption{$P{\omega_i}=1/3$での識別結果}
  \label{tbl:prob-discrimination-1}
\end{table}

\begin{table}[htbp]
  \begin{tabular}{cccc}
    % test point 1 & test point 2 & test point 3 & test point 4 \\
    $(1, 2, 1)^T$ & $(5, 3, 2)^T$ & $(0, 0, 0)^T$ & $(1, 0, 0)^T$ \\
    $\omega_2$ & $\omega_2$ & $\omega_1$ & $\omega_1$
  \end{tabular}
  \caption{$P{\omega_1}=0.8$, $P{\omega_2}=P{\omega_3}=0.1$ での識別結果}
  \label{tbl:prob-discrimination-2}
\end{table}

